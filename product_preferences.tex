\PassOptionsToPackage{unicode=true}{hyperref} % options for packages loaded elsewhere
\PassOptionsToPackage{hyphens}{url}
%
\documentclass[]{article}
\usepackage{lmodern}
\usepackage{amssymb,amsmath}
\usepackage{ifxetex,ifluatex}
\usepackage{fixltx2e} % provides \textsubscript
\ifnum 0\ifxetex 1\fi\ifluatex 1\fi=0 % if pdftex
  \usepackage[T1]{fontenc}
  \usepackage[utf8]{inputenc}
  \usepackage{textcomp} % provides euro and other symbols
\else % if luatex or xelatex
  \usepackage{unicode-math}
  \defaultfontfeatures{Ligatures=TeX,Scale=MatchLowercase}
\fi
% use upquote if available, for straight quotes in verbatim environments
\IfFileExists{upquote.sty}{\usepackage{upquote}}{}
% use microtype if available
\IfFileExists{microtype.sty}{%
\usepackage[]{microtype}
\UseMicrotypeSet[protrusion]{basicmath} % disable protrusion for tt fonts
}{}
\IfFileExists{parskip.sty}{%
\usepackage{parskip}
}{% else
\setlength{\parindent}{0pt}
\setlength{\parskip}{6pt plus 2pt minus 1pt}
}
\usepackage{hyperref}
\hypersetup{
            pdfborder={0 0 0},
            breaklinks=true}
\urlstyle{same}  % don't use monospace font for urls
\usepackage{longtable,booktabs}
% Fix footnotes in tables (requires footnote package)
\IfFileExists{footnote.sty}{\usepackage{footnote}\makesavenoteenv{longtable}}{}
\setlength{\emergencystretch}{3em}  % prevent overfull lines
\providecommand{\tightlist}{%
  \setlength{\itemsep}{0pt}\setlength{\parskip}{0pt}}
\setcounter{secnumdepth}{0}
% Redefines (sub)paragraphs to behave more like sections
\ifx\paragraph\undefined\else
\let\oldparagraph\paragraph
\renewcommand{\paragraph}[1]{\oldparagraph{#1}\mbox{}}
\fi
\ifx\subparagraph\undefined\else
\let\oldsubparagraph\subparagraph
\renewcommand{\subparagraph}[1]{\oldsubparagraph{#1}\mbox{}}
\fi

% set default figure placement to htbp
\makeatletter
\def\fps@figure{htbp}
\makeatother

\title{Modeling Aggregate Demand for Products and Services based on Consumer Preferences for Product Features}
\author{Larry D. Lee Jr.}
\date{August 9, 2024}

\begin{document}

\maketitle

\begin{abstract}
This article presents a model of aggregate demand for products
and services based on consumer preferences for product features.
Organizations offering services and products can use this model to
prioritize feature development and service improvement initiatives by
estimating the potential impact of product changes on aggregate demand.
\end{abstract}

\hypertarget{introduction}{%
\subsection{Introduction}\label{introduction}}

In this article, we present a mathematical model of aggregate demand for
products and services based on consumer preferences for product
features. We show how to use this model to estimate the potential impact
that changes to product features will have on consumer demand. Using
these techniques, organizations can prioritize product development and
service improvement efforts.

\hypertarget{the-model}{%
\subsection{The Model}\label{the-model}}

We start by assuming that there is a set of products \(A_i\) and
potential consumers in a marketplace. We characterize the products using
a set of common feature dimensions, such as weight, cost, and
performance. Thus, every product is associated with a vector listing its
feature values \([x_{j,i}]\). Every potential consumer assigns a
different weight to each feature dimension. These weights are
represented by the consumer's preference vector \([k_{l,j}]\). Consumers
assign a quality score to each product
\(s_{l,i} = \sum_{j=0}^n k_{l,j} x_{j,i}\) that is a linear combination
of the product's feature values and the consumer's feature weights.

We assume that, for every feature dimension \(j\), consumer preference
weights \(k_{l,j}\) are normally distributed and that preferences across
dimensions are independent.\footnote{In general, preferences will not be
  independent across feature dimensions. For example, the importance
  that a consumer assigns to the weight of a product might correlate
  with the weight that they assign to its size. Thus, in practice, we
  will typically need to perform factor analysis such as Principal
  Component Analysis (PCA) to identify a set of independent feature
  dimensions.} Because the quality scores are linear combinations of
independent normally distributed variables, the quality scores for each
product will also be normally distributed.

Let \(\mu_j\) and \(\sigma_j\) represent the mean and standard deviation
of weights that consumers give to feature dimension \(j\). The
distribution of consumer quality scores \(s_i\) for a product \(A_i\)
will equal \footnote{Note that the second argument to \(\mathcal{N}\)
  gives the distribution's variance.}\footnote{Soch, J. (2021, June 2).
  Proof: Linear combination of independent normal random variables. The
  Book of Statistical Proofs.
  \url{https://statproofbook.github.io/P/norm-lincomb.html} }

\begin{align}
s_i \sim \mathcal{N} (\sum_{j=0}^n x_{j,i}\ \mu_j, \sum_{j=0}^n x_{j,i}^2\ \sigma_j^2).
\end{align}

When presented with two products \(A_i\) and \(A_j\), a potential
consumer will prefer \(A_i\) over \(A_j\) when \(s_i > s_j\). If
\(s_i = s_j\), the consumer will be indifferent between \(A_i\) and
\(A_j\) and will choose one of them randomly with probability
\(\frac{1}{2}\).

Let the quality scores \(s_i\) for product \(A_i\) be normally
distributed with mean \(\mu_i\) and standard deviation \(\sigma_i\).
Additionally, let the quality scores \(s_j\) for another product \(A_j\)
be normally distributed with mean \(\mu_j\) and standard deviation
\(\sigma_j\). We can calculate the probability \(p_{i,j}\) that a
consumer will choose \(A_i\) over \(A_j\) as

\begin{align}
p_{i,j} = \int \int_0^{\infty} \varphi(\frac{s - \mu_j}{\sigma_j})\ \varphi (\frac{s + \delta - \mu_i}{\sigma_i})\ d\delta\ ds.
\end{align}

(1) is equivalent to

\begin{align}
p_{i,j} = \Phi (\frac{\mu_i - \mu_j}{\sqrt{\sigma_i^2 + \sigma_j^2}}).
\end{align}

The derivation is presented in Appendix 1.

Let \(p\) represent the probability that a randomly selected consumer
chooses either product \(A_i\) or \(A_j\). Then, if there are \(n\)
consumers, in \(np\) times, a consumer will choose either \(A_i\) or
\(A_j\). Let \(n_i\) and \(n_j\) represent the number of times that a
consumer chooses \(A_i\) and \(A_j\) respectively.

\begin{align*}
np\ p_{i, j} &= n_i\\
np\ p_{j, i} &= n_j.
\end{align*}

Equating both equations gives:

\begin{align*}
np = \frac{n_i}{p_{i,j}} = \frac{n_j}{p_{j,i}}
\end{align*}

Rearranging gives the relative frequency \(k_{i,j}\) with which
consumers choose \(A_i\) over \(A_j\).

\begin{align*}
k_{i,j} = \frac{p_{i,j}}{p_{j,i}} = \frac{n_i}{n_j}
\end{align*}

Let \(p_i\) represent the absolute probability that a consumer will
choose product \(A_i\). Then, assuming that every consumer chooses a
product, we can form the equation:

\begin{align*}
p_0 + p_1 + \cdots + p_n = 1.
\end{align*}

We can substitute the likelihood ratios into this equation to form:

\begin{align*}
& p_0 + k_{1,0} p_0 + \cdots + k_{n,0} p_0 &= 1\\
& \cdots\\
& k_{0,n} p_n + k_{1,n} p_n + \cdots + p_n &= 1\\
\end{align*}

We can easily solve these equations to calculate the absolute
probability that a consumer will choose a each product. For example:

\begin{align*}
p_0 = \frac{1}{1 + k_{1,0} + \cdots + k_{n,0}}.
\end{align*}

Thus, we can calculate the proportion of consumers who will choose each
product offered within a marketplace based on their preferences for
product features. Using these equations, we can calculate the effect
that feature changes will have on aggregate demand for various products.
\pagebreak
\hypertarget{example-use}{%
\subsection{Example Use}\label{example-use}}

Imagine that we want to model the aggregate demand for three computer
processors. Each processor is characterized by myriad features however,
we determine that four factors, price, power, area, and performance,
explain most consumer preference. We conduct a series of surveys and
determine that the weights assigned to these factors are independent and
normally distributed.

\begin{longtable}[]{@{}llll@{}}
\toprule
\(j\) & Factor & Mean \(\mu\) & Standard Deviation
\(\sigma\)\tabularnewline
\midrule
\endhead
0 & Price & -3 & 1\tabularnewline
1 & Power & -2 & 3\tabularnewline
2 & Performance & 4 & 3\tabularnewline
3 & Area & -1 & 2\tabularnewline
\bottomrule
\end{longtable}

\begin{quote}
\textbf{Table 1: Example Factor Weights for Computer Processors.} This
table presents example factor weights to illustrate how product features
influence consumer product preferences.
\end{quote}

We then measure three computer processors along each of these four
dimensions and produce the following measurements.

\begin{longtable}[]{@{}lllll@{}}
\toprule
Product & Price \(x_{0,i}\) & Power \(x_{1,i}\) & Performance
\(x_{2,i}\) & Area \(x_{3,i}\)\tabularnewline
\midrule
\endhead
\(A_0\) & 1 & 1 & 1 & 1\tabularnewline
\(A_1\) & 10 & 3 & 5 & 3\tabularnewline
\(A_2\) & 2 & 2 & 3 & 2\tabularnewline
\bottomrule
\end{longtable}

\begin{quote}
\textbf{Table 2: Example Product Factor Values.} This table presents
example factor measurements for three hypothetical computer processors
to illustrate how the aggregate demand model can be used.
\end{quote}

Based on these factor values, we can use (1) to calculate the score
distribution for each product.

\begin{longtable}[]{@{}lll@{}}
\toprule
Product & Mean Score \(\mu\) & Score Standard Deviation
\(\sigma\)\tabularnewline
\midrule
\endhead
\(A_0\) & -2 & 4.7958\tabularnewline
\(A_1\) & -19 & 21.0238\tabularnewline
\(A_2\) & 0 & 11.7047\tabularnewline
\bottomrule
\end{longtable}

\begin{quote}
\textbf{Table 3: Example Product Quality Scores.} This table presents
probability distribution parameters for example product quality scores
(utilities).
\end{quote}

Once we have derived the distributions for the product quality scores,
we can calculate the probability \(p_{i,j}\) that a randomly selected
consumer will choose one product over another for each pair of products
\(A_i\) and \(A_j\) using (3). We can record these probabilities in a
matrix like the following

\begin{align*}
P := \begin{bmatrix}
0.5000 & 0.7848 & 0.4372\\
0.2152 & 0.5000 & 0.2149\\
0.5628 & 0.7851 & 0.5000\\
\end{bmatrix}
\end{align*}.

Dividing along the diagonals, we can calculate the likelihood ratios
\(k_{i,j}\) for each pair of products

\begin{align*}
K := \begin{bmatrix}
1.0000 & 3.6459 & 0.7768\\
0.2743 & 1.0000 & 0.2737\\
1.2874 & 3.6538 & 1.0000\\
\end{bmatrix}
\end{align*}.

Adding elements within each column, we can calculate the probability
that a randomly selected consumer will choose each product

\begin{align*}
A_0 &= \frac{1}{ 1.0000 + 0.2743 + 1.2874 } = 0.3904\\
A_1 &= \frac{1}{ 3.6459 + 1.0000 + 3.6538 } = 0.1205\\
A_2 &= \frac{1}{ 0.7768 + 0.2737 + 1.0000 } = 0.4877\\
\end{align*}

However, we can do more than calculate relative market share for each
product. We can go further and calculate the impact that changes to
product features will have on market share. For example, we can use this
model to calculate the price that maximizes the gross earnings for the
company that produces the second product \(A_1\).

Let's assume that the market size consists of 10 million units and that
the following function is used to convert price scores into actual
prices

\begin{align*}
price = 5 (e^{priceScore/10} - 1)
\end{align*}

Then each product has the following gross earnings

\begin{longtable}[]{@{}ll@{}}
\toprule
Product & Gross Earnings\tabularnewline
\midrule
\endhead
\(A_0\) & \$2,052,795.88\tabularnewline
\(A_1\) & \$10,351,453.05\tabularnewline
\(A_2\) & \$5,398,845.97\tabularnewline
\bottomrule
\end{longtable}

\begin{quote}
\textbf{Table 4: Example Product Gross Earnings.} This table presents
example gross earnings for a set of hypothetical products.
\end{quote}

However, our model indicates that the company that manufactures \(A_1\)
will actually maximize their gross earnings if they can reduce their
price from \$8.59 to \$6.56. Doing so will increase their market share to 16.06\% and increase their
gross earnings to \$10,529,874.61.

\hypertarget{conclusion}{%
\subsection{Conclusion}\label{conclusion}}

In this article, we have presented a mathematical model of aggregate
product demand based on consumer preferences for product features. We
have shown how to use this model to calculate the effect that product
changes will have on consumer demand. This knowledge can be used to
prioritize product development initiatives and to support planning.

While we have focused on products, the same methods can be used to model
demand for services. Additionally, while our examples have assumed that
the total number of units sold within a market is fixed, we can easily
extend the model to represent variable demand by introducing a "null"
product.

In practice, the primary difficulty using this model, as is often the
case, lies in parameterization. Unfortunately, it is not possible to
"run the equations backwards" and infer the input variables such as
product feature scores from output variables such as market share. In
general, there are many different product "configurations" that will
produce the same market share divisions. As a result, the methods
introduced in this article are best used for qualitative and approximate
modeling. Input variables such as product scores, and consumer weights,
can be approximated using consumer surveys and expert judgement.

Lastly, we hope that our mathematical solutions to the equations are valuable.
While are confident that these equations have been presented and solved
in other contexts, we were unable to find solutions to them in our
readings. Hence, we hope that this article will make their derivation
and proofs easier to find.


\hypertarget{appendix-1-solving-the-choice-equation}{%
\subsection{Appendix 1: Solving the Choice
Equation}\label{appendix-1-solving-the-choice-equation}}

Assume that a population of consumers choose between two products
\(A_i\) and \(A_j\). Every consumer assigns a quality score \(s_i\) and
\(s_j\) to products \(A_i\) and \(A_j\) respectively. These quality
scores are normally distributed with means \(\mu_i\) and \(\mu_j\) and
standard deviations \(\sigma_i\) and \(\sigma_j\) respectively. A
consumer will choose \(A_i\) over \(A_j\) if \(s_i > s_j\). If
\(s_i = s_j\) the consumer will be indifferent and will choose one of
them randomly with probability \(\frac{1}{2}\). The probability that a
randomly selected consumer will choose \(A_i\) over \(A_j\) is given by

\begin{align}
p_{i,j} = \int \int_0^{\infty} \varphi(\frac{s - \mu_j}{\sigma_j})\ \varphi (\frac{s + \delta - \mu_i}{\sigma_i})\ d\delta\ ds
\end{align}

In this section, we will show that (4) equals

\begin{align}
\Phi (\frac{\mu_i - \mu_j}{\sqrt{\sigma_i^2 + \sigma_j^2}}).
\end{align}

To prove this equality, we will calculate the Fourier transform for (4)
and (5) and show that they are the same.\footnote{Whenever two functions
  have identical Fourier transforms, we know that they are equal. }

Recall that the Fourier transform for the Normal Cumulative Density
function (CDF) is

\begin{align}
k(\xi) := \int \Phi (\frac{x_0 - \mu}{\sigma})\ e^{-2 \pi i \xi x_0} d x_0 = \frac{e^{-2 \pi i \xi \mu} e^{-2(\pi \xi \sigma)^2}}{2 \pi i \xi} + \frac{1}{2}\delta(\xi)
\end{align}

where \(\delta\) is the Dirac Delta function. In Appendix 2, we show how
we can derive (6). From this equation, we can use the Inverse Fourier
Transform to express the Normal CDF as

\begin{align}
\Phi (\frac{x_0 - \mu}{\sigma}) = \int \frac{e^{2 \pi i \xi (x_0 - \mu)} e^{-2(\pi \xi \sigma)^2}}{2 \pi i \xi} d \xi + \frac{1}{2}.
\end{align}

From (6) we see that the Inverse Fourier Transform for (4) equals

\begin{align}
\Phi (\frac{\mu_i - \mu_j}{\sqrt{\sigma_i^2 + \sigma_j^2}}) = \int \frac{e^{2 \pi i \xi (\mu_i - \mu_j)} e^{-2(\pi \xi \sqrt{\sigma_i^2 + \sigma_j^2})^2}}{2 \pi i \xi} d \xi + \frac{1}{2}.
\end{align}

We will show that the Fourier transform for (4) has the same form as
(8). Let \(k(\xi)\) represent the Fourier transform for
\(\varphi (\frac{s - \mu_j}{\sigma_j})\) and replace this term in (3)
with its Fourier transform:

\begin{align*}
p_{i,j} &= \int \int_0^{\infty} \varphi(\frac{s - \mu_j}{\sigma_j})\ \varphi (\frac{s + \delta - \mu_i}{\sigma_i})\ d\delta\ ds\\
&= \int \int_0^{\infty} (\int k(\xi) e^{2 \pi i \xi s}\ d\xi)\ \varphi (\frac{s + \delta - \mu_i}{\sigma_i})\ d\delta\ ds\\
&\hspace{2em}(substitute\ the\ Inverse\ Fourier\ transform)\\
&= \int k(\xi) \int_0^{\infty} \int e^{2 \pi i \xi s}\ \varphi (\frac{s + \delta - \mu_i}{\sigma_i})\ ds\ d\delta\ d\xi\\
&= \int k(\xi) \int_0^{\infty} e^{2 \pi i \xi (\mu_i - \delta)} e^{-2(\pi \xi \sigma_i)^2}\ d\delta\ d\xi \hspace{3em}\\
&\hspace{2em}(similar\ to\ the\ derivation\ for\ the\ Fourier\ transform\ of\ the\ Normal\ PDF) \\
&= \int k(\xi) e^{2 \pi i \xi \mu_i}  e^{-2(\pi \xi \sigma_i)^2}\int_0^{\infty} e^{-2 \pi i \xi \delta}\ d\delta\ d\xi\\
&= \int \frac{k(\xi) e^{2 \pi i \xi \mu_i}  e^{-2(\pi \xi \sigma_i)^2}}{2 \pi i \xi} d\xi\\
&= \int \frac{e^{-2 \pi i \xi \mu_j}\ e^{-2(\pi \xi \sigma_j)^2} e^{2 \pi i \xi \mu_i}  e^{-2(\pi \xi \sigma_i)^2}}{2 \pi i \xi} d\xi\hspace{3em}\\
&\hspace{2em}(expand\ k(\xi))\\
&= \int \frac{e^{2 \pi i \xi (\mu_i - \mu_j)}\ e^{-2(\pi \xi \sqrt{\sigma_i^2 + \sigma_j^2})^2}}{2 \pi i \xi} d\xi.
\end{align*}

All that remains is to set the constant of integration, which in this
instance is \(\frac{1}{2}\). Thus, we see that the Fourier inverse
transform for (4) is equivalent to the inverse transform for (5).

\hypertarget{appendix-2-the-fourier-transforms-of-the-normal-pdf-and-cdf}{%
\subsection{Appendix 2: The Fourier Transforms of the Normal PDF and
CDF}\label{appendix-2-the-fourier-transforms-of-the-normal-pdf-and-cdf}}

In this section, we will show how we can calculate the Fourier Transform
for the Normal Probability Density Function (PDF) and Cumulative Density
Function (CDF).

We can calculate the Fourier Transform for the PDF by expanding terms
and completing the square as follows

\begin{align*}
k(\xi) &:= \int \varphi (\frac{x - \mu}{\sigma}) e^{-2 \pi i \xi x} dx \\
&= \frac{1}{\sigma \sqrt{2 \pi}} \int e^{-\frac{(x - \mu)^2}{s \sigma^2}} e^{-2 \pi i \xi x} dx\\
&= \frac{1}{\sigma \sqrt{2 \pi}} \int e^{-\frac{1}{2 \sigma^2}(x^2 - 2 \mu x + \mu^2) - 2 \pi i \xi x} dx\\
&= \frac{1}{\sigma \sqrt{2 \pi}} \int e^{-\frac{1}{2 \sigma^2}(x^2 - 2 \mu x + \mu^2 + 4 \pi i \xi \sigma^2 x)} dx\\
&= \frac{1}{\sigma \sqrt{2 \pi}} \int e^{-\frac{1}{2 \sigma^2}(x^2 - 2x (\mu - 2 \pi i \xi \sigma^2) + (\mu - 2 \pi i \xi \sigma^2)^2 - (\mu - 2 \pi i \xi \sigma^2)^2 + \mu^2)} dx \hspace{3em}\\
&\hspace{2em}(complete\ the\ square)\\
&= \frac{1}{\sigma \sqrt{2 \pi}} \int e^{-\frac{1}{2 \sigma^2}((x - (\mu - 2 \pi i \xi \sigma^2))^2 - (\mu - 2 \pi i \xi \sigma^2)^2 + \mu^2)} dx\\
&= \frac{1}{\sigma \sqrt{2 \pi}} \int e^{-\frac{1}{2 \sigma^2}((x - (\mu - 2 \pi i \xi \sigma^2))^2 + 4 \mu \pi i \xi \sigma^2 + 4(\pi \xi \sigma^2)^2)} dx\\
&= \frac{1}{\sigma \sqrt{2 \pi}} \int e^{-\frac{1}{2 \sigma^2}((x - (\mu - 2 \pi i \xi \sigma^2))^2}e^{-2 \mu \pi i \xi - 2(\pi \xi \sigma)^2)} dx\\
&= e^{-2 \mu \pi i \xi - 2(\pi \xi \sigma)^2} \int \varphi(\frac{x - (\mu - 2 \pi i \xi \sigma^2)}{\sigma})\ dx\\
&= e^{-2 \mu \pi i \xi - 2(\pi \xi \sigma)^2}.
\end{align*}

We can then use this equation to calculate the Inverse Fourier transform
for the Normal PDF.

\begin{align*}
\varphi (\frac{x - \mu}{\sigma}) &= \int k(\xi) e^{2\pi i \xi x}\ d\xi\\
&= \int e^{-2 \mu \pi i \xi - 2(\pi \xi \sigma)^2} e^{2\pi i \xi x}\ d\xi\\
&= \int e^{2 \pi i \xi (x - \mu) - 2(\pi \xi \sigma)^2}\ d\xi\\
\end{align*}

We can actually simplify this expression further. To do so, note that
the integral is quasi symmetric about \(\xi = 0\). Rewrite the integral
so that it spans from 0 to \(\infty\) and move the expression for both
\(+\xi\) and \(- \xi\) into it. Next, expand the complex exponential
into sines and cosines. The sine terms will cancel out. The result will
be

\begin{align*}
\varphi (\frac{x - \mu}{\sigma}) = 2 \int_0^{\infty} cos (2 \pi \xi (x - \mu)) e^{-2(\pi \xi \sigma)^2} d\xi 
\end{align*}

From the above calculations, we see that the Inverse Fourier transform
for (3) must equal

\begin{align}
\varphi(\frac{\mu_i - \mu_j}{\sqrt{\sigma_i^2 + \sigma_j^2}}) = \int e^{2 \pi i \xi (\mu_i - \mu_j)}\ e^{-2(\pi \xi \sqrt{\sigma_i^2 + \sigma_j^2})^2} d\xi.
\end{align}

We can use the Integration Property of the Fourier Transform to
calculate the Fourier Transform for the Normal CDF. Given a continuous
function \(f (x)\), the Integration Property says

\begin{align}
\mathcal{F} (\int_{-\infty}^{x_0} f (x) dx) = \frac{\mathcal{F} (f(x))}{2 \pi i \xi} + c\ \delta(\xi)
\end{align}

Where \(\mathcal{F}\) denotes the Fourier Transform operator, \(\delta\)
represents the Dirac delta function, and \(c\) is a constant that
depends on \(f\). In the standard proof the integral in (10) is
represented by a convolution of \(f\) and the step-function. We then use
the Convolution Property of the Fourier Transform to express the Fourier
Transform as the product of the Fourier Transform of \(f\) and the step
function. The Fourier Transform of the step function reduces to a sum
involving the Dirac Delta function. A detailed derivation can be found
in \footnote{Bevelacqua, P. (n.d.). \emph{The Integration Property of
  the Fourier Transform}. The Fourier Transform.com.
  \url{https://www.thefouriertransform.com/transform/integration.php}}.

Applying the Integration Property of the Fourier Transform to the Normal
CDF gives

\begin{align}\Phi (\frac{x_0 - \mu}{\sigma}) = \int \frac{e^{2 \pi i \xi (x_0 - \mu)} e^{-2(\pi \xi \sigma)^2}}{2 \pi i \xi} d \xi + \frac{1}{2}.\end{align}

\end{document}
